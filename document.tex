%%This is a very basic article template.
%%There is just one section and two subsections.
\documentclass[12pt, a4paper]{article}

\usepackage[utf8]{inputenc} % Kodowanie UTF8
\usepackage[MeX]{polski} % Wsparcie dla PL

\begin{document}


\section{Branża hotelarska}

\subsection{Wstęp}

Branża hotelarska jest bardzo różnorodna, począwszy od ogromnych sieci
hotelarskich o zasięgu światowym, które obsługują miliony klientów dziennie i zatrudniają
 jeszcze więcej osób przez średniej wielkości hotele, które działają
 samodzielnie, aż po małe, rodzinne biznesy oferujące swoje usługi
 w turystycznych miejscowościach.
 
  Wszystkie mają jedną wspólną cechę,
 świadczą usługi noclegowe dla swoich klientów. Jest to podstawowy element
 działalności hotelarskiej. Cała reszta usług ma sprawiać, aby klient był
 bardziej zadowolony z pobytu, ponieważ to on jest tutaj najważniejszy.
 
  Dochodzimy tutaj do prostego wniosku, że znakomita
 większość spraw związanych z prowadzeniem hotelu jest taka sama jak dla innych
 biznesów i tyczą się ich te same problemy. W innych biznesach, tam gdzie mamy
 zamówienia, tutaj występują rezerwacje, a towar jest pobytem w hotelu. Istnieje
 cały szereg analogii i dopiero na tym poziomie można zobaczyć czym tak naprawdę
 różni się prowadzenie hotelu od chociażby sklepu wysyłkowego.
 
 Istnieje jednak pewna subtelna różnica pomiędzy branżą hotelarską, a innymi
 brażnami. Mianowicie w tym segmencie biznesu jesteśmy bardzo, ale to bardzo
 zależni od poziomu zadowolenia klienta i zależy nam na tym, aby poziom ten był
 jak najwyższy. Jeden nieusatysfakcjonowany klient to strata kilkunastu
 innych potencjalnych klientów, którym ów klient odradzi pobyt w naszym hotelu.
 Jest to szczególnie dotkliwe dla tych mniejszych i średnich hoteli, gdzie nie
 można pozwolić sobię na taką stratę.

\subsection{Klasyfikacja}
W pierwszym spojrzeniu na hotel, jego organizację i hotelowych gości można go
zaklasyfikować do jednej z dwóch kategorii:
\begin{itemize}
  \item Turystyczny
  \item Biznesowy 
\end{itemize}

Hotele, które należą do jednego albo drugiego profilu biznesowego różnią się pod
bardzo wieloma aspektami. Najważniejsze z nich to lokalizacja, wystrój
zarówno wewnętrzny jak i zewnętrzny, rodzaj klientów oraz świadczone usługi
w hotelu. Typowe również jest to, że dla hoteli turystycznych weekendy są zazwyczaj
 droższe niż pobyty w środku tygodnia. Sytuacja jest zupełnie odwrotna dla grupy
 biznesowej. Charakterystycznym dla hoteli turystycznych jest to, że obłożenie
 hotelu jest wysokie zazwyczaj tylko w tzw. sezonie turystycznym, a przez resztę
 roku utrzymuje się na niższym poziomie.
 
 W klasie hoteli o profilu biznesowym średnie obłożenie będzie zazwyczaj wyższe
 niż dla tych z grupy turystycznej, ale na pewno nie jest to reguła, która się
  sprawdza zawsze. Tutaj również istnieje sezon, w którym obłożenie hotelu
  osiąga swój szczyt. Pomijając specjalne okoliczności takie jak EURO2012, takim
  sezonem jest sezon konferecyjny, który zaczyna się na przełomie marca i kwietnia.
  
  Dla hoteli z profilu biznesowego następuje 
 jeszcze jeden podział ze względu na politykę wynajmowania albo łóżek albo pokoi.
   
 Wytłumaczenie różnic w politykach:
 
 W jednym hotelu możemy wynająć pokój np.
 3 osobowy i cena pozostanie taka sama jeśli będziemy tam sami albo w dwie osoby albo w pięć jeśli jest to hotel,
  który działa zgodnie z polityką wynajmowania pokoi. Inna sytuacja nastąpi w hotelach, które wynajmują łóżka,
   cena będzie się różniła dość znacznie. Takie hotele również
   zazwyczaj nie godzą się na to aby w pokoju nocowało więcej osób niż liczba łóżek. 
   Hotele klasy turystycznej zazwyczaj prowadzą politykę wynajmowania łóżek, a hotele klasy biznesowej wynajmowania pokoi.

\subsection{Rezerwacja}
Elementem wspólnym dla każdego hotelu jest rezerwacja.
Rezerwacja to dokument, który zapowiada pewne zdarzenie
z przyszłości jakim jest pobyt w hotelu w określonych dniach,
pokoju, warunkach i osobach, które ten pobyt będą odbywać.

Niestety nie ma utartego schematu jednej rezerwacji i tego jakie informacje
powinna zawierać. Można wyszczególnić pewien zbiór informacji, który zawiera się
na każdej rezerwacji, a przynajmniej powinien się zawierać. Jednak istnieją
hotele, które pozwalają na wprowadzanie wielu dodatkowych informacji do
rezerwacji. Zazwyczaj są to hotele wysokiej klasy, a informacje które możemy
podać odnoszą się np. do typu poduszki jaki chcemy, rodzaju łóżka lub innych
upodobań. Więcej informacji o tym co może zawierać informacji znajdziesz w
sekcji \ref{informacje_na_rezerwacji}


\subsubsection{Gwarantowana rezerwacja}
Rezerwacja zazwyczaj wymaga pewnej gwarancji ze strony rezerwującego, ponieważ
hotel zobowiązuje się, aby pokój, na który prowadzona jest rezerwacja
był w danym czasie dostępny. Jeśli kto inny chciałby wtedy dokonać rezerwacji 
zakładając, że był to ostatni wolny pokój to wtedy stracimy klienta jeśli 
rezerwujący się wycofa albo nie przyjdzie. Potrzebny jest zatem pewien  
mechanizm zabezpieczający hotel przez potencjalnymi stratami. 

Typowy mechanizm gwarantowanej rezerwacji dla klientów indywidualnych przy dokonywaniu 
jej online polega na podaniu numeru karty kredytowej, która w przypadku 
nieodwołania rezerwacji przed określonym terminem przez hotel zostanie obciążona pewną karą 
również określoną przez hotel. Dla klientów indywidualnych niektóre hotele 
umożliwiają dokonywanie rezerwacji niegwarantowanej. Polega to na tym, 
że można dokonać rezerwacji z datą przyjazdu na dzisiaj bez podawania 
numeru karty, ale jest ona trzymana do określonej godziny, 
zazwyczaj 16 czasu lokalnego, po której to już nie mamy pewności 
czy pokój nie zostanie wynajęty komuś innemu. Możliwość dokonania
niegwarantowanej rezerwacji jest świadoczna tylko przez niektóre hotele.
Zazwyczaj każdy woli być zabezpieczony. 

Niegwarantowana rezerwacja to dość wygodne rozwiązanie szczególnie w krajach jak
Polska, gdzie posiadanie karty kredytowej nie jest jeszcze tak popularne. Warto zaznaczyć, że sposób gwarancji rezerwacji 
nie ma nic wspólnego z metodą płatności za pobyt. Przykładowo gwarantować
rezerwacje można kartą kredytową, a płacić gotówką albo przelewem albo innym
środkiem płatności akceptowanym przez hotel.


\subsubsection{Informacje na rezerwacji}
\label{informacje_na_rezerwacji}
Rezerwacja przechowuje dość sporo informacji. Standardowa porcja informacji, 
które są wymagane aby dokonać rezerwacji jest następująca:
\begin{itemize}
  \item Data przyjazdu,
  \item data wyjazdu,
  \item rodzaj pokoju,
  \item dane rezerwującego
  \begin{itemize}
    \item imię,
    \item nazwisko, 
    \item adres, 
    \item kontakt jak e-mail, numer telefonu.
  \end{itemize} 
\end{itemize}

Oprócz takich podstawowych informacji bardzo często możemy podać także wiele 
innych informacji. Często umożliwiają to hotele o wysokim standardzie. Niektóre
z możliwych informacji to np. preferencje:
\begin{itemize}
  \item czy pokój dla palących czy niepalących,
  \item umiejscowienie pokoju,
  \begin{itemize}
    \item niskie piętro,
    \item wysokie piętro,
    \item blisko windy,
  \end{itemize}
  \item widok,
  \item rodzaj poduszki.
\end{itemize}

Podać można także takie informacje jak:
\begin{itemize}
  \item kod promocyjny - często hotele biznesowe, a szczególnie duże sieci
  hotelarskie o światowym zasięgu oferują programy lojalnościowe, w których
  zbiera się punkty, które poźniej można wymienić np. na voucher na darmowy
  pobyt w hotelu. Kod promocyjny odnosi się właśnie do tych nagród,
  \item przynależność do grupy,
  \item przynależność do korporacji,
  \item przynależność do agenta biura podróżniczego,
  \item planowana godzina przyjazdu,
  \item informacje dodatkowe dla recepcji jak specjalne życzenia bądź uwagi
\end{itemize}

Inne informacje zależą tylko od inwencji własnej osoby odpowiedzialnej za hotel,
która będzie chciała zrobić tak żeby klienci czuli się jak najlepiej. Część z nich 
taka jak odpowiedni kod grupy lub korporacji ma związek z typem klientów hotelu.
Typ klientów został opisany w sekcji \ref{rodzaje_klientow}

\subsubsection{Sposób dokonywania rezerwacji}
Rezerwacji można dokonywać:
\begin{itemize}
  \item online – uzupełniając formularz na stronie hotelu
  \item telefonicznie – podając dane pracownikowi recepcji. Wtedy przez 
  całą procedurę przeprowadzi nas pracownik recepcji. Wszystkie dane powtarzane
  są dwa razy w celu sprawdzenia ich poprawności przez obie strony.
  \item osobiście – jeśli chcemy dokonać osobiście rezerwacji w hotelu w
  recepcji to możemy to zrobić, po czym dostajemy albo klucz do 
  pokoju – jeśli chcieliśmy mieć pokój na teraz albo potwierdzenie rezerwacji, 
  które jest dokumentem zawierającym podstawowe dane rezerwacji oraz 
  numer rezerwacji
\end{itemize}

\subsubsection{Odwołanie rezerwacji}
Odwołanie rezerwacji jest zazwyczaj możliwe i nie wiąże się z karą jeśli 
rezerwacja zostanie odwołana przed określonym terminem. Zazwyczaj, ponieważ
niektóre hotele całkowicie zabraniają odwoływania rezerwacji zatem zawsze wiąże
się to z utrzymaniem karą. Przykładem są, niektóre hotele w sieci Marriott.
Te hotele, które umożliwiają odwołanie rezerwacji wymuszają pewne
zasady na których to odwołanie się odbywa zazwyczaj jest to określona godzina,
do której możemy odwołać rezerwację. Przykładowy termin to godzina 16 czasu
lokalnego hotelu w planowanym dniu przyjazdu. Może to być także zupełnie inny termin arbitralnie wybrany przez osobę za to odpowiedzialna. 
  Po określonym terminie odwołanie jest możliwe jeśli hotel ma taką politykę. 
  Zazwyczaj odwołanie rezerwacji przeprowadza się telefonicznie dlatego także to w gestii 
  pracownika hotelu jest ustalenie co zrobić, czy odwołać rezerwację bez żadnej kary nawet po terminie
   czy jeśli klient coś kręci to nie dać się nabrać i odpowiednio zareagować. 
  Ważne jest tutaj odpowiednie przeszkolenie pracowników i ich doświadczenie.
  
  Należy pamiętać o tym, że jeśli mówimy, że nie możemy przyjechać w danym dniu,
  ponieważ lot nam się opóźnił to taka informacja jest łatwa do sprawdzenia
  przez pracownika hotelu i jeśli pracownik ma podejrzenie, że coś jest nie tak
  to właśnie zostanie to sprawdzone.

Jeśli nie pojawimy się w hotelu w dniu rezerwacji to typowym mechanizmem jest
to, że w następnym dniu rezerwacje które nie zostały zrealizowane oznaczane są
jako NO SHOW. Dalsza procedura zależy od hotelu albo obarczą karą tego kto
gwarantował rezerwację albo będą próbowali skontaktować się z rezerwującym. Nie
ma tutaj mechanizmu unierwalnego. Należy jednak pamiętać, że karanie klienta
jest rozwiązaniem, które przynosi korzyści tylko krótkoterminowo, a hotel
powinien być zainteresowany zyskami długoterminowymi dlatego też warto aby
istniała procedura opierająca się na wyjaśnienie przyczyny, dialog z klientem i
wspólne porozumienie.

\subsubsection{Relokacja rezerwacji}
Relokacja rezerwacji polega na przesunięciu w czasie daty 
przyjazdu lub/i wyjazdu. Jest to możliwe do zrobienia. Warto jednak pamiętać o
tym że o ile sama relokacja nic nie kosztuje to cena rezerwacji może 
wzrosnąć, jeśli przesuwamy ją na termin, w którym cena pobytu jest po prostu 
większa. Prosta zasada to nowy termin, nowa wycena. Relokacja jest możliwa 
wtedy gdy mamy wolne pokoje w terminie, na który klient chce przesunąć
rezerację.

\subsection{Rodzaje klientów}
\label{rodzaje_klientow}
Gości hotelowych można podzielić na parę kategorii. Przynależność do danej
kategorii ma wpływ przede wszystkim na cenę pobytu jak również na inne 
dodatkowe usługi oferowane przez hotel. 

Podział wygląda następująco:
\begin{itemize}
  \item klient indywidualny,
  \item klient korporacyjny,
  \item klient grupowy,
  \item klient związany z agentem biura podróżniczego,
  \item klient stały.
\end{itemize}

\subsubsection{Klient indywidualny}
Klient indywidualny jest to po prostu zwykła osoba z ulicy, która zazwyczaj
płaci najwyższą stawkę za pobyt i inne usługi dodatkowe jakie oferuje hotel.

\subsubsection{Klient korporacyjny}
Klient, który należy do korporacji, która ma z hotelem podpisaną umowę.
Zazwyczaj wygląda to tak, że korporacja zobowiązuje się do wykorzystania 
pewnej liczby pobytów w roku co powoduje iż cena jest odpowiednio niższa.
Wszystkie szczegóły reguluje umowa.

\subsubsection{Klient grupowy}
W hotelach czasem organizowane są konferencje, bale bądź wystawy. Uczestnik
takiego wydarzenia należy do grupy.

Zdarza się także, że na czas pobytu grupy w hotelu w recepcji ustawiane 
jest dodatkowe stanowisko, które obsługuje tylko i wyłącznie członków danej 
grupy. Taka sytuacja zazwyczaj ma miejsce w większych hotelach.

Ceny dla takich grupowych klientów, podobnie jak z klientami korporacyjnymi,
reguluje umowa pomiędzy hotelem, a organizatorem danego wydarzenia.

\subsubsection{Klient związany z agentem biura podróżniczego}
Przy wyszukiwaniu miejsca na spędzenie wakacji udajemy się zazwyczaj do biura
podróżniczego, które oferuje nam kompleksowo zaplanowane wakacje.

Między innymi elementami układanki znajduje się hotel, w którym odbędziemy pobyt
i właśnie taki klient jest klasyfikowany do tej grupy.

Również w tym przypadku występuje umowa pomiędzy agencją turystyczną, a hotelem.

\subsubsection{Klient stały}
Istnieją tacy klienci którzy po prostu mieszkają w hotelach. Przykładem może być
Pan Krauze i Hotel Marriott.  

Taki klient ma gwarancje tego, że zawsze dostaną
swój ulubiony pokój więc zazwyczaj jest on dla nich trzymany wolny. Klient stały
ma szereg udogodnień zarówno w usługach jak i płatnościach np. przekładanie
płatności lub łączenie ich za wiele pobytów.

Tutaj istnieje ogromna dowolność w obsługiwaniu takiego klienta i nie należy
tego ograniczać żadnymi ramami.

\subsubsection{Grupy w kontekście rezerwacji}
Członkowie grupy lub klienci korporacyjni mają wszystko albo opłacone 
albo płacą sami za siebie, wtedy jest to cenowo nadal mniej niż dla klientów 
indywidualnych. W przypadku gdy wszystko jest opłacone po prostu
wystawiana jest faktura na odpowiednią firmę, z którą związana jest dana
grupa. Wszystko zależy od tego jak sporządzona zostanie umowa pomiędzy hotelem,
a innym podmiotem gospodarczym.

Każdy uczestnik grupy ma swoją własną rezerwację. To, że należy do grupy jest 
odnotowane w informacjach zawartych w rezerwacji. To samo tyczy się klientów 
należących do innej kategorii. Rezerwacja jest wystawiana na konkretne osoby, 
a nie firmy z którymi podpisana jest umowa.

\subsection{Długość pobytów}
Wiadomym jest, że istnieje dowolność w długości pobytów, choć czasem mogą być
pewne różnice w cenie za noc jeśli spędzimy jedną noc lub trzy pod rząd.

To co jednak nie jest tak bardzo rozpowszechnione to to, że w hotelach,
zazwyczaj w tych z klasy biznesowej, możliwe jest odbycie krótkoterminowego
pobytu.
\subsubsection{Krótkoterminowy pobyt}
Jeśli potrzebujemy się tylko odświeżyć albo spędzić kilka godzin w klientem na
rozmowie albo wystąpił inny realny scenariusz to warto skorzystać z opcji jaką
jest pobyt do godziny po południowej.
Przykładowo możemy wynająć pokój w taki sposób, że musimy go opuścić np. do
godziny 16 i wtedy nie zapłacimy standardowo za noc, tylko odpowiednio mniej.
Nie jest to przelicznik godzinowy. Po prostu można odbyć pobyt do południa i
koszt takiego pobytu będzie na pewno mniejszy niż byśmy wynajeli pokój na noc.

\subsection{Konferencje}
Wydawać by się mogło, że głównym źródłem dochodu dla hotelu jest oferowanie
pokoi. Jednak często jest tak, że spora część dochodu dla hotelu ma źródło z
przyjmowania konferencji. Oczywiście nadal w grę wchodzą pokoje dla uczestników
konferencji, ale sytuacja całościwo wygląda inaczej dlatego warto o niej
wspomnieć.

Są hotele, które utrzymują się praktycznie tylko z organizowania konferencji, a
pobyty dla osób indywidualnych są bardzo małą częścią zysku hotelu. Przykładem
takiego hotelu jest hotel Gromada w Warszawie, który posiada największą bazę sal
konferencyjnch w Warszawie, a przynajmniej jedną z większych.

Jak już wspomniałem o salach konferencyjnych to jest to właśnie nieodzwony
element konferencji. Oprócz tego trzeba ulokować jeszcze uczestników
konferencji w pokojach. Zazwyczaj jest tak, że pokoje dla uczestników
konferencji są obok siebie. Nie jest to reguła.

W niektórych hotelach, na czas konferencji lub innego grupowego wydarzenia
uruchamiane jest specjalne stanowisko w recepcji, które obsługuje tylko
uczestników danej konferencji. Przykładem takiego hotelu może być hotel Marriott
w Warszawie. Wszystko zależy od wielkości grupy i szczegółów umowy jaka jest
podpisana między hotelem, a organizatorem konferencji. Nie mniej jednak
możliwość osobnego stanowiska istnieje. 

Przypomnę jeszcze, że każdy uczestnik konferencji ma własną rezerwację.

\subsection{Usługi}
Usługi, pod tym pojęciem kryje się wiele. Sekcję tą poświęcam na opisanie tych
najczęściej zamawianych oraz nakreślam jakie inne usługi mogą być oferowane
abstrahując od takich jak trufle na złotej tacy podane do łóżka.

\subsubsection{Definicja usługi}
Usługa to pewna korzyść, z której odpłatnie lub nie może skorzystać klient
podczas swojego pobytu. Pod to pojęcie można by podciągnąć bardzo wiele rzeczy.
Przykładowo korzystanie z basenu hotelowego, siłowni lub spa. Chciałbym się
jednak skupić nie na usłudze w sensie bogatej oferty hotelowej, a na tych
usługach, które wymagają specjalnego traktowania przez obsługę hotelu innego niż
dopisania tego do rachunku.
\subsubsection{Przykłady usług}
Przykładem takiej usługi, która może być przydatna dla zwykłego człowieka jest
budzenie. Zamawiamy w repecji budzenie na godzinę taką i taką i mamy nadzieje,
że zostaniemy o tej godzinie obudzeni. Czy to nastąpi poprzez telefon czy
stukanie do drzwi to już zależy od wewnętrznej ogranizacji.

Innym przykładem usługi, może być dostarczenie klientowi np. świeżych elementów
wysposażenia łazienkowego takiego jak ręcznik.

Kolejna usługa to np. dostarczenie do pokoju koja oraz kuwety dla naszego
pupila.

Nie sposób jest wymienić wszystkich możliwych usług, ponieważ wszystko zależy od
wyobraźni klienta i tego czy obsługa hotelu może i chce sprostać zadaniu. A jak
wiemy powinniśmy robić wszystko, aby zadowolenie klienta było jak największe.
Oczywiście wszystko ma swoją cenę, a cenę za niestandardowe rzeczy ponosi
klient.

\subsubsection{Klasyfikacja usług}
Można by zatem spróbować sklasyfikować usługi biorąc pod uwagę to czy system
informatyczny może pomóc w ich zrealizowaniu czy też nie. Czynnika ludzkiego
nie sposób jest wyeliminować w spełnianiu zachcianek klienta, tak więc tutaj
odgrywa on rolę szczególną.

System na pewno może pomóc w zapamiętaniu kto zamawiał budzenie i na kiedy i
odpowiednio przypomnieć o tym obsłudze hotelu lub samemu wykonać telefon jeśli
dysponujemy odpowiednią infrastrukturą. Zatem na pewno warto uwzględnić
podsystem notyfikacji. Natomiast w jaki sposób system może pomóc w przypadku gdy
klient zapyta o coś, czego nawet teraz w momencie pisania nie jestem w stanie
określić. Na pewno warto zapewnić możliwość dołączenia pewnych notatek do
profilu klienta np. informujących o tym, że klient posiada kota albo psa. Wtedy
następnym razem gdy klient nas odwiedzi będzie zaskoczony, że potrzebne
wyposażenie ma już na miejscu. Można sobie łatwo wyobrazić zadowolenie klienta z
faktu, że ktoś o nim pamiętał. 

\end{document}
